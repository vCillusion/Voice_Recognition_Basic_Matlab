
% This LaTeX was auto-generated from an M-file by MATLAB.
% To make changes, update the M-file and republish this document.

\documentclass{article}
\usepackage{graphicx}
\usepackage{color}

\sloppy
\definecolor{lightgray}{gray}{0.5}
\setlength{\parindent}{0pt}

\begin{document}

    
    
\subsection*{Contents}

\begin{itemize}
\setlength{\itemsep}{-1ex}
   \item Project: Voice Recognition and Identification system
   \item Main Function Voice Recognition
   \item 1: Human speaker recognition
   \item 2: Technical data of samples
   \item 3: linear and logarithmic power spectrum plot
   \item 4: Plots for different values for N
   \item 5: Mel Space
   \item 6: Modified spectrum
   \item 7: 2D plot of accustic vectors
   \item 8: Plot of the 2D trained VQ codewords
   \item 9: Recognition rate of the computer
   \item 9.1 Loading from Test Folder for Comparison
   \item 9.2 Comparing one by one data from TRAIN FOLDER
   \item 9.3 Note: Test with other speech files
   \item 10: Voice Recognition by letting user enter into database and then compare
   \item 10.1 Add a new sound from microphone
   \item 10.2 Voice Recognition from microphone
   \item 10.3 Database Info
   \item 10.4 Delete database
   \item blockFrames Function
   \item MFCC Function
   \item VQLBG Vector quantization using the Linde-Buzo-Gray algorithm
\end{itemize}


\subsection*{Project: Voice Recognition and Identification system}

\begin{par}
By  Mahima Garg, Omar Razi, Supriya Phutela, Vaibhav Kapoor, Varun Chopra
\end{par} \vspace{1em}
\begin{verbatim}
%--------------------------------------------------------------------------
\end{verbatim}


\subsection*{Main Function Voice Recognition}

\begin{verbatim}
function []=voicerecognition()
% For clear screen
clc;
% Ronaldo is a variable used to directly set the minimum distance for
% speech recognition.
ronaldo=10;
% st, st1, st2, st3,  will be used for filenames related purposes reducing
% code redundancy.
char st; char st1; char st2; char st3;
disp('Project: Voice Recognition and Identification system');
disp('By  Mahima Garg & Omar Razi & Supriya Phutela & Vaibhav Kapoor & Varun Chopra ');
disp(' ');
pause(0.5);
disp('LOADING ');
pause(1);
disp('... ');
pause(1);
disp('... ');
pause(1);
disp('... ');
pause(1);
disp('... ');
msgbox('We have tried to make this project as user friendly as possible. We Hope you Appreciate and Enjoy !!!','Voice Recognition and Identification System');

% Preallocating array
str = {8}; fstr = {8}; nbtr = {8};
ste = {8}; fste = {8}; nbte = {8};
ctr = {8}; dtr={8};
cte = {8}; dte={8};
data = {8,4};
code = {8};

for i = 1:8

    % Read audio data from train folder for performing operations
    st=strcat('train\s',num2str(i),'.wav');
    [s1 fs1 nb1]=wavread(st);
    str{i} = s1; fstr{i} = fs1; nbtr{i} = nb1;

    % Read audio data from test folder for performing operations
    st = strcat('test\s',num2str(i),'.wav');
    [st1 fst1 nbt1] = wavread(st);
    ste{i} = st1; fste{i} = fst1; nbte{i} = nbt1;

    % Compute MFCC of the audio data to be used in Speech Processing for Train
    % Folder
    ctr{i} = mfcc(str{i},fstr{i});

    % Compute MFCC of the audio data to be used in Speech Processing for Test
    % Folder
    cte{i} = mfcc(ste{i},fste{i});

    % Compute Vector Quantization of the audio data to be used in Speech
    % Processing for Train Folder
    dtr{i} = vqlbg(ctr{i},16);

    % Compute Vector Quantization of the audio data to be used in Speech
    % Processing for Test Folder
    dte{i} = vqlbg(cte{i},16);
end

% For making Choice
ch=0;
poss=11;
while ch~=poss

    ch=menu('Speaker Recognition System','1: Human speaker recognition',...
        '2: Technical data of samples',...
        '3: Power Spectrum','4: Power Spectrum with different M and N',...
        '5: Mel-Spaced Filter Bank',...
        '6: Spectrum before and after Mel-Frequency wrapping',...
        '7: 2D plot of acoustic vectors',...
        '8: Plot of VQ codewords','9: Recognition rate of the computer',...
        '10: Test with other speech files','11: Exit');
    disp('                                                                  ');
    %----------------------------------------------------------------------
\end{verbatim}


\subsection*{1: Human speaker recognition}

\begin{verbatim}
    % Play each sound file in the TRAIN folder. Can you distinguish the voices
    % of those eight speakers? Now play each sound in the TEST folder in a
    % random order without looking at the file name (pretending that you do not
    % know the speaker) and try to identify the speaker using your knowledge of
    % their voices that you just learned from the TRAIN folder. This is exactly
    % what the computer will do in our system.
    % What is your (human performance) recognition rate ?
    % Record this result so that it can be used later to be compared with the
    % computer's performance of our system. Both of us seem to be unable to
    % recognise random people just by listening at their voice. Our success
    % rates for the provided samples were 1 person out of 8 each.
    %
    % However, for the samples we used in question 10, we were easily able to
    % recognise each speaker. This is probably because we knew all the persons.
    % We also realized that we did not identify speakers by the frequencies
    % they use to talk, but rather by other characteristics, like accent,
    % speed, etc.

    if ch==1
        disp('>  1: Human speaker recognition');
        disp('Play each sound file in the TRAIN folder.');
        disp('Can you distinguish the voices of those eight speakers?');
        disp('Now play each sound in the TEST folder in a random order without looking at the file name ');
        disp('and try to identify the speaker using your knowledge of their voices that you have just heard,');
        disp('from the TRAIN folder. This is exactly what the computer will do in our system.');
        disp('                                                                                                  ');
        disp('                                                                                                   ');
        disp('All of us seem to be unable to recognise random people just by listening to their voice. ');
        disp('We also realize that we do not identify speakers by the frequencies with which they use to talk,  ');
        disp('but rather by other characteristics, like accent, speed, etc.');
        pause(1);

        ch2=0;
        while ch2~=4
            ch2=menu('Select Folder','Train','Test','User','Exit');

            if ch2==1

                ch3=0;
                while ch3~=9
                    ch3=menu('Train :','Signal 1','Signal 2','Signal 3',...
                        'Signal 4','Signal 5','Signal 6','Signal 7','Signal 8','Exit');
                    if ch3~=9
                        p=audioplayer(str{ch3},fstr{ch3},nbtr{ch3});
                        play(p);
                    end
                end
            end

            if ch2==2

                ch3=0;
                while ch3~=9
                    ch3=menu('Test :','Signal 1','Signal 2','Signal 3',...
                        'Signal 4','Signal 5','Signal 6','Signal 7','Signal 8','Exit');
                    if ch3~=9
                        p=audioplayer(ste{ch3},fste{ch3},nbte{ch3});
                        play(p);
                    end
                end
                close all;
            end

            if ch2==3

                if (exist('sound_database.dat','file')==2)
                    load('sound_database.dat','-mat');

                    ch32=0;
                    while ch32 ~=2
                        ch32=menu('Database Information','Database','Exit');

                        if ch32==1
                            st=strcat('Sound Database has : #',num2str(sound_number),'words. Enter a database number : #');
                            prompt = {st};
                            dlg_title = 'Database Information';
                            num_lines = 1;
                            def = {'1'};
                            options.Resize='on';
                            options.WindowStyle='normal';
                            options.Interpreter='tex';
                            an = inputdlg(prompt,dlg_title,num_lines,def);
                            an=cell2mat(an);
                            a = str2double(an);

                            if (isempty(an))

                            else

                                if (a <= sound_number)
                                    st=strcat('u',num2str(an));
                                    [s fs nb]=wavread(st);
                                    p=audioplayer(s,fs,nb);
                                    play(p);

                                else
                                    warndlg('Invalid Word ','Warning');
                                end
                            end
                        end
                        close all;
                    end

                else
                    warndlg('Database is empty.',' Warning ')
                end
            end

        end
    end
    %----------------------------------------------------------------------
\end{verbatim}


\subsection*{2: Technical data of samples}

\begin{verbatim}
    % Read a sound file into Matlab. Check it by playing the sound file in
    % Matlab using the function: sound. What is the sampling rate? What is the
    % highest frequency that the recorded sound can capture with fidelity? With
    % that sampling rate, how many msecs of actual speech are contained in a
    % block of 256 samples?
    %
    % Plot the signal to view it in the time domain. It should be obvious that
    % the raw data in the time domain has a great amount of data and for this
    % reason it is difficult to analyse the voice characteristic. So the
    % purpose of this step (speech feature extraction) should be clear now!
    %
    % Intermediate steps that follows:
    %
    % Frames blocking phase
    % To obtain a matrix M containing all the frames, we used the following script:
    % That way we obtain the 256 x 129 matrix M.
    %
    % Windowing phase
    % We create the Hamming matrix and transform the matrix M into the new
    % matrix M2, where the column vectors of M2 are the original frame vectors
    % transformed by the Hamming filter.
    %
    % FFT Phase
    % We create a new matrix M3 where the column vectors are the FFTs of the
    % column vectors of M2.

    if ch==2
        disp('> 2: Technical data of samples');

        ch23=0;
        while ch23~=4
            ch23=menu('Select Folder','Train','Test','User','Exit');

            if ch23==1
                poss2=9;
                ch2=0;
                while ch2~=poss2
                    ch2=menu('Technical data of samples for :','Signal 1','Signal 2','Signal 3',...
                        'Signal 4','Signal 5','Signal 6','Signal 7','Signal 8','Exit');
                    if ch2~=9
                        t = 0:1/fstr{ch2}:(length(str{ch2}) - 1)/fstr{ch2};
                        plot(t, str{ch2}), axis([0, (length(str{ch2}) - 1)/fstr{ch2} -0.4 0.5]);
                        st=sprintf('Plot of signal s%d.wav',ch2);
                        title(st);
                        xlabel('Time [s]');
                        ylabel('Amplitude (normalized)')
                    end
                end
                close all
            end

            if ch23==2
                poss2=9;
                ch2=0;
                while ch2~=poss2
                    ch2=menu('Technical data of samples for :','Signal 1','Signal 2','Signal 3',...
                        'Signal 4','Signal 5','Signal 6','Signal 7','Signal 8','Exit');
                    if ch2~=9
                        t = 0:1/fste{ch2}:(length(ste{ch2}) - 1)/fste{ch2};
                        plot(t, ste{ch2}), axis([0, (length(ste{ch2}) - 1)/fste{ch2} -0.4 0.5]);
                        st=sprintf('Plot of signal s%d.wav',ch2);
                        title(st);
                        xlabel('Time [s]');
                        ylabel('Amplitude (normalized)')
                    end
                end
                close all
            end

            if ch23==3

                if (exist('sound_database.dat','file')==2)
                    load('sound_database.dat','-mat');

                    ch32=0;
                    while ch32 ~=2
                        ch32=menu('Database Information','Database','Exit');

                        if ch32==1
                            st=strcat('Sound Database has : #',num2str(sound_number),'words. Enter a database number : #');
                            prompt = {st};
                            dlg_title = 'Database Information';
                            num_lines = 1;
                            def = {'1'};
                            options.Resize='on';
                            options.WindowStyle='normal';
                            options.Interpreter='tex';
                            an = inputdlg(prompt,dlg_title,num_lines,def);
                            an=cell2mat(an);
                            a = str2double(an);

                            if (isempty(an))

                            else

                                if (a <= sound_number)
                                    st=strcat('u',num2str(an));
                                    [s fs]=wavread(st);
                                    t = 0:1/fs:(length(s) - 1)/fs;
                                    plot(t, s), axis([0, (length(s) - 1)/fs -0.4 0.5]);
                                    st=sprintf('Plot of signal %s',st);
                                    title(st);
                                    xlabel('Time [s]');
                                    ylabel('Amplitude (normalized)')

                                else
                                    warndlg('Invalid Word ','Warning');
                                end
                            end

                        end
                    end
                    close all;

                else
                    warndlg('Database is empty.',' Warning ')
                end
            end

        end
    end
    %----------------------------------------------------------------------
\end{verbatim}


\subsection*{3: linear and logarithmic power spectrum plot}

\begin{verbatim}
    % After successfully running the preceding process, what is the
    % interpretation you can make of the result obtained? Compute the power
    % spectrum and plot it out using the imagesc command. Note that it is
    % better to view the power spectrum on the log scale. Locate the region in
    % the plot that contains most of the energy. Translate this location into
    % the actual ranges in time (msec) and frequency (in Hz) of the input
    % speech signal.
    %
    % As stated before, the columns matrix M3 contain the frames of the
    % original signal, filtered by the Hamming filter and transformed with the
    % FFT. The elements of M3 are complex numbers and symmetrical because FFT
    % was used to transform the data.
    % Each column in M3 is a power spectrum representation of the original signal.
    % To plot the power spectrum, we take the absolute values of the matrix
    % elements. Since the spectrum is symmetric, we only plot half of it. Note
    % that we plot simultaneously with the linear and the logarithmic spectrum
    % scale.
    %
    % The result obtained is the plot in this section.
    % In this plot, the areas containing the highest level of energy are
    % displayed in red. As we can see on the plot, the red area is located
    % between 0.3 and 0.7 seconds. The plot also shows that most of the energy
    % is concentrated in the lower frequencies (between 50 Hz and 1 kHz).

    if ch==3
        M = 100;
        N = 256;
        disp('> 3: Power Spectrum Plot');
        disp('                                                      ');
        disp('>Linear and Logarithmic spectrum plot');

        ch23=0;
        while ch23~=4
            ch23=menu('Select Folder','Train','Test','User','Exit');

            if ch23==1
                poss3=9;
                ch3=0;
                while ch3~=poss3
                    ch3=menu('Linear and Logarithmic Power Spectrum Plot for : ','Signal 1','Signal 2','Signal 3',...
                        'Signal 4','Signal 5','Signal 6','Signal 7','Signal 8','Exit');
                    if ch3~=9
                        % 3 (linear)
                        frames = blockFrames(str{ch3}, fstr{ch3}, M, N);
                        t = N / 2;
                        tm = length(str{ch3}) / fstr{ch3};
                        subplot(121);
                        imagesc([0 tm], [0 fstr{ch3}/2], abs(frames(1:t, :)).^2), axis xy;
                        title('Power Spectrum (M = 100, N = 256)');
                        xlabel('Time [s]');
                        ylabel('Frequency [Hz]');
                        colorbar;
                        % 3 (logarithmic)
                        subplot(122);
                        imagesc([0 tm], [0 fstr{ch3}/2], 20 * log10(abs(frames(1:t, :)).^2)), axis xy;
                        title('Logarithmic Power Spectrum (M = 100, N = 256)');
                        xlabel('Time [s]');
                        ylabel('Frequency [Hz]');
                        colorbar;
                        % D=get(gcf,'Position');
                        % set(gcf,'Position',round([D(1)*.5 D(2)*.5 D(3)*2 D(4)*1.3]))
                    end
                end
                close all
            end

            if ch23==2
                poss3=9;
                ch3=0;
                while ch3~=poss3
                    ch3=menu('Linear and Logarithmic Power Spectrum Plot for : ','Signal 1','Signal 2','Signal 3',...
                        'Signal 4','Signal 5','Signal 6','Signal 7','Signal 8','Exit');
                    if ch3~=9
                        % 3 (linear)
                        frames = blockFrames(ste{ch3}, fste{ch3}, M, N);
                        t = N / 2;
                        tm = length(ste{ch3}) / fste{ch3};
                        subplot(121);
                        imagesc([0 tm], [0 fste{ch3}/2], abs(frames(1:t, :)).^2), axis xy;
                        title('Power Spectrum (M = 100, N = 256)');
                        xlabel('Time [s]');
                        ylabel('Frequency [Hz]');
                        colorbar;
                        % 3 (logarithmic)
                        subplot(122);
                        imagesc([0 tm], [0 fste{ch3}/2], 20 * log10(abs(frames(1:t, :)).^2)), axis xy;
                        title('Logarithmic Power Spectrum (M = 100, N = 256)');
                        xlabel('Time [s]');
                        ylabel('Frequency [Hz]');
                        colorbar;
                        % D=get(gcf,'Position');
                        % set(gcf,'Position',round([D(1)*.5 D(2)*.5 D(3)*2 D(4)*1.3]))
                    end
                end
                close all;
            end

            if ch23==3

                if (exist('sound_database.dat','file')==2)
                    load('sound_database.dat','-mat');

                    ch32=0;
                    while ch32 ~=2
                        ch32=menu('Database Information','Database','Exit');

                        if ch32==1
                            st=strcat('Sound Database has : #',num2str(sound_number),'words. Enter a database number : #');
                            prompt = {st};
                            dlg_title = 'Database Information';
                            num_lines = 1;
                            def = {'1'};
                            options.Resize='on';
                            options.WindowStyle='normal';
                            options.Interpreter='tex';
                            an = inputdlg(prompt,dlg_title,num_lines,def);
                            an=cell2mat(an);
                            a = str2double(an);

                            if (isempty(an))

                            else

                                if (a <= sound_number)
                                st=strcat('u',num2str(an));
                                [s fs]=wavread(st);
                                frames = blockFrames(s, fs, M, N);
                                t = N / 2;
                                tm = length(s) / fs;
                                subplot(121);
                                imagesc([0 tm], [0 fs/2], abs(frames(1:t, :)).^2), axis xy;
                                title('Power Spectrum (M = 100, N = 256)');
                                xlabel('Time [s]');
                                ylabel('Frequency [Hz]');
                                colorbar;
                                %Question 3 (logarithmic)
                                subplot(122);
                                imagesc([0 tm], [0 fs/2], 20 * log10(abs(frames(1:t, :)).^2)), axis xy;
                                title('Logarithmic Power Spectrum (M = 100, N = 256)');
                                xlabel('Time [s]');
                                ylabel('Frequency [Hz]');
                                colorbar;

                                else
                                    warndlg('Invalid Word ','Warning');
                                end
                            end
                        end
                    end
                    close all;

                else
                    warndlg('Database is empty.',' Warning ')
                end
            end

        end
    end
    %----------------------------------------------------------------------
\end{verbatim}


\subsection*{4: Plots for different values for N}

\begin{verbatim}
    % Compute and plot the power spectrum of a speech file using different
    % frames sizes: for example N = 128, 256 and 512. In each case, set the
    % frame increement M to be about N/3. Can you describe and explain the
    % differences between these spectra ?
    %
    % For N = 128 we have a high resolution of time. Furthermore each frame
    % lasts a very short period of time. This result shows that the signal for
    % a frame doesn't change its nature (i.e. it will be for the same vowel or
    % consonant). On the other hand, there are only 65 distinct frequencies
    % samples. This means that we have a poor frequency resolution.
    %
    % For N = 256 we have a compro mise between the resolution in time and the
    % frequency resolution.
    % For N = 512 we have an excellent frequency resolution (256 different
    % values) but there are lesser frames, meaning that the resolution in
    % time is strongly reduced.
    % It seems that a value of 256 for N is an acceptable compromise.
    % Furthermore the number of frames is relatively small, which will reduce
    % computing time.

    if ch==4
        disp('>  4: Plots for different values for M and N');
        lN = [128 256 512];

        ch23=0;
        while ch23~=4
            ch23=menu('Select Folder','Train','Test','User','Exit');

            if ch23==1
                poss3=9;
                ch3=0;
                while ch3~=poss3
                    ch3=menu('Plots for different values of M and N for :','Signal 1','Signal 2','Signal 3',...
                        'Signal 4','Signal 5','Signal 6','Signal 7','Signal 8','Exit');
                    if ch3~=9
                        u=220;
                        for i = 1:length(lN)
                            N = lN(i);
                            M = round(N / 3);
                            frames = blockFrames(str{ch3}, fstr{ch3}, M, N);
                            t = N / 2;
                            tm = length(str{ch3}) / fstr{ch3};
                            temp = size(frames);
                            nbframes = temp(2);
                            u=u+1;
                            subplot(u)
                            imagesc([0 tm], [0 fstr{ch3}/2], 20 * log10(abs(frames(1:t, :)).^2)), axis xy;
                            title(sprintf('Power Spectrum (M = %i, N = %i, frames = %i)', M, N, nbframes));
                            xlabel('Time [s]');
                            ylabel('Frequency [Hz]');
                            colorbar
                        end
                        % D=get(gcf,'Position');
                        % set(gcf,'Position',round([D(1)*.5 D(2)*.5 D(3)*1.5 D(4)*1.5]))
                    end
                end
                close all
            end

            if ch23==2
                poss3=9;
                ch3=0;
                while ch3~=poss3
                    ch3=menu('Plots for different values of M and N for :','Signal 1','Signal 2','Signal 3',...
                        'Signal 4','Signal 5','Signal 6','Signal 7','Signal 8','Exit');
                    if ch3~=9
                        u=220;
                        for i = 1:length(lN)
                            N = lN(i);
                            M = round(N / 3);
                            frames = blockFrames(ste{ch3}, fste{ch3}, M, N);
                            t = N / 2;
                            tm = length(ste{ch3}) / fste{ch3};
                            temp = size(frames);
                            nbframes = temp(2);
                            u=u+1;
                            subplot(u)
                            imagesc([0 tm], [0 fste{ch3}/2], 20 * log10(abs(frames(1:t, :)).^2)), axis xy;
                            title(sprintf('Power Spectrum (M = %i, N = %i, frames = %i)', M, N, nbframes));
                            xlabel('Time [s]');
                            ylabel('Frequency [Hz]');
                            colorbar
                        end
                        % D=get(gcf,'Position');
                        % set(gcf,'Position',round([D(1)*.5 D(2)*.5 D(3)*1.5 D(4)*1.5]))
                    end
                end
                close all;
            end

            if ch23==3

                if (exist('sound_database.dat','file')==2)
                    load('sound_database.dat','-mat');

                    ch32=0;
                    while ch32 ~=2
                        ch32=menu('Database Information','Database','Exit');

                        if ch32==1
                            st=strcat('Sound Database has : #',num2str(sound_number),'words. Enter a database number : #');
                            prompt = {st};
                            dlg_title = 'Database Information';
                            num_lines = 1;
                            def = {'1'};
                            options.Resize='on';
                            options.WindowStyle='normal';
                            options.Interpreter='tex';
                            an = inputdlg(prompt,dlg_title,num_lines,def);
                            an=cell2mat(an);
                            a = str2double(an);

                            if (isempty(an))

                            else

                                if (a <= sound_number)
                                    st=strcat('u',num2str(an));
                                    [s fs]=wavread(st);
                                    u=220;
                                    for i = 1:length(lN)
                                        N = lN(i);
                                        M = round(N / 3);
                                        frames = blockFrames(s, fs, M, N);
                                        t = N / 2;
                                        tm = length(s) / fs;
                                        temp = size(frames);
                                        nbframes = temp(2);
                                        u=u+1;
                                        subplot(u)
                                        imagesc([0 tm], [0 fs/2], 20 * log10(abs(frames(1:t, :)).^2)), axis xy;
                                        title(sprintf('Power Spectrum (M = %i, N = %i, frames = %i)', M, N, nbframes));
                                        xlabel('Time [s]');
                                        ylabel('Frequency [Hz]');
                                        colorbar
                                    end
                                else
                                    warndlg('Invalid Word ','Warning');
                                end
                            end
                        end
                     end
                    close all;

                else
                    warndlg('Database is empty.',' Warning ')
                end
            end
        end
    end
    %----------------------------------------------------------------------
\end{verbatim}


\subsection*{5: Mel Space}

\begin{verbatim}
    % Type 'help melfb' at the Matlab prompt for more information about this
    % function. Follow the guidelines to plot out the mel-spaced filter bank.
    % What is the behaviour of this filter bank? Compare it with the
    % theoretical part. To plot this filter bank, we use melfb command.
    % This filter bank behaves like a succession of histograms on the spectrum.
    % Each filter of the filter bank has a triangular frequency response. It
    % quantifies the zone of the frequency spectrum.
    %
    % The filter bank is used to transform the spectrum of a signal into a
    % representation which reflects more closely the behaviour of the human
    % ear. As the human ear (or the associated neurons) favours low frequencies
    % for analysing speech, the filters are denser for the lower frequencies.
    % To mimic the human ear, the filters are linearly distributed for low
    % frequencies (below 1kHz). For higher frequencies (above 1 kHz) the
    % distribution of the filters is logarithmic.
    % However we calculated 20 filters (instead of 12).

    if ch==5
        disp('> 5: Mel Space');
        disp('                                                                 ');
        disp('Mel Space is function of sampling rate and since all signals ');
        disp('are recorded at same sampling rate so they have same Mel Space.');

        ch23=0;
        while ch23~=4
            ch23=menu('Select Folder','Train','Test','User','Exit');

            if ch23==1
                poss3=9;
                ch3=0;
                while ch3~=poss3
                    ch3=menu('Mel Space for :','Signal 1','Signal 2','Signal 3',...
                        'Signal 4','Signal 5','Signal 6','Signal 7','Signal 8','Exit');
                    if ch3~=9
                        plot(linspace(0, (fstr{ch3}/2), 129), (melfb(20, 256, fstr{ch3})));
                        title('Mel-Spaced Filterbank');
                        xlabel('Frequency [Hz]');
                    end
                end
                close all
            end


            if ch23==2
                poss3=9;
                ch3=0;
                while ch3~=poss3
                    ch3=menu('Mel Space for :','Signal 1','Signal 2','Signal 3',...
                        'Signal 4','Signal 5','Signal 6','Signal 7','Signal 8','Exit');
                    if ch3~=9
                        plot(linspace(0, (fste{ch3}/2), 129), (melfb(20, 256, fste{ch3})));
                        title('Mel-Spaced Filterbank');
                        xlabel('Frequency [Hz]');
                    end
                end
                close all;
            end

            if ch23==3

                if (exist('sound_database.dat','file')==2)
                    load('sound_database.dat','-mat');

                    ch32=0;
                    while ch32 ~=2
                        ch32=menu('Database Information','Database','Exit');

                        if ch32==1
                            st=strcat('Sound Database has : #',num2str(sound_number),'words. Enter a database number : #');
                            prompt = {st};
                            dlg_title = 'Database Information';
                            num_lines = 1;
                            def = {'1'};
                            options.Resize='on';
                            options.WindowStyle='normal';
                            options.Interpreter='tex';
                            an = inputdlg(prompt,dlg_title,num_lines,def);
                            an=cell2mat(an);
                            a=str2double(an);

                            if (isempty(an))

                            else

                                if (a <= sound_number)
                                    st=strcat('u',num2str(an));
                                    [s fs]=wavread(st);
                                    plot(linspace(0, (fs/2), 129), (melfb(20, 256, fs)));
                                    title('Mel-Spaced Filterbank');
                                    xlabel('Frequency [Hz]');
                                else
                                    warndlg('Invalid Word ','Warning');
                                end
                            end
                        end
                    end
                    close all;

                else
                    warndlg('Database is empty.',' Warning ')
                end
            end

        end
    end
    %----------------------------------------------------------------------
\end{verbatim}


\subsection*{6: Modified spectrum}

\begin{verbatim}
    % Compute and plot the spectrum of a speech file before and after the mel-
    % frequency wrapping step. Describe and explain the impact of the melfb
    % program.
    %
    % As we can see in the first plot, most of the information is contained in
    % the lower frequencies. This information has been extracted and amplified
    % in the second plot. The second plot therefore shows the main
    % characteristics of the speech signal.
    % Note that the transformation produced an acoustic vector of 20 dimensions.

    if ch==6
        disp('> 6: Modified spectrum');
        disp('                                                              ');
        disp('Spectrum before and after Mel-Frequency wrapping');
        M = 100;
        N = 256;
        n2 = 1 + floor(N / 2);

        ch23=0;
        while ch23~=4
            ch23=menu('Select Folder','Train','Test','User','Exit');

            if ch23==1
                poss3=9;
                ch3=0;
                while ch3~=poss3
                    ch3=menu('Mel Space for :','Signal 1','Signal 2','Signal 3',...
                        'Signal 4','Signal 5','Signal 6','Signal 7','Signal 8','Exit');
                    if ch3~=9
                        frames = blockFrames(str{ch3}, fstr{ch3}, M, N);
                        m = melfb(20, N, fstr{ch3});
                        z = m * abs(frames(1:n2, :)).^2;
                        tm = length(str{ch3}) / fstr{ch3};
                        subplot(121)
                        imagesc([0 tm], [0 fstr{ch3}/2], abs(frames(1:n2, :)).^2), axis xy;
                        title('Power Spectrum unmodified');
                        xlabel('Time [s]');
                        ylabel('Frequency [Hz]');
                        colorbar;
                        subplot(122)
                        imagesc([0 tm], [0 20], z), axis xy;
                        title('Power Spectrum modified through Mel Cepstrum filter');
                        xlabel('Time [s]');
                        ylabel('Number of Filter in Filter Bank');
                        % colorbar;D=get(gcf,'Position');
                        % set(gcf,'Position',[0 D(2) D(3)/2 D(4)])
                    end
                end
                close all
            end


            if ch23==2
                poss3=9;
                ch3=0;
                while ch3~=poss3
                    ch3=menu('Mel Space for :','Signal 1','Signal 2','Signal 3',...
                        'Signal 4','Signal 5','Signal 6','Signal 7','Signal 8','Exit');
                    if ch3~=9
                        frames = blockFrames(str{ch3}, fstr{ch3}, M, N);
                        m = melfb(20, N, fstr{ch3});
                        z = m * abs(frames(1:n2, :)).^2;
                        tm = length(str{ch3}) / fstr{ch3};
                        subplot(121)
                        imagesc([0 tm], [0 fstr{ch3}/2], abs(frames(1:n2, :)).^2), axis xy;
                        title('Power Spectrum unmodified');
                        xlabel('Time [s]');
                        ylabel('Frequency [Hz]');
                        colorbar;
                        subplot(122)
                        imagesc([0 tm], [0 20], z), axis xy;
                        title('Power Spectrum modified through Mel Cepstrum filter');
                        xlabel('Time [s]');
                        ylabel('Number of Filter in Filter Bank');
                        % colorbar;D=get(gcf,'Position');
                        % set(gcf,'Position',[0 D(2) D(3)/2 D(4)])
                    end
                end
                close all;
            end

            if ch23==3

                if (exist('sound_database.dat','file')==2)
                    load('sound_database.dat','-mat');

                    ch32=0;
                    while ch32 ~=2
                        ch32=menu('Database Information','Database','Exit');

                        if ch32==1
                            st=strcat('Sound Database has : #',num2str(sound_number),'words. Enter a database number : #');
                            prompt = {st};
                            dlg_title = 'Database Information';
                            num_lines = 1;
                            def = {'1'};
                            options.Resize='on';
                            options.WindowStyle='normal';
                            options.Interpreter='tex';
                            an = inputdlg(prompt,dlg_title,num_lines,def);
                            an=cell2mat(an);
                            a=str2double(an);

                            if (isempty(an))

                            else

                                if (a <= sound_number)
                                    st=strcat('u',num2str(an));
                                    [s fs]=wavread(st);
                                    frames = blockFrames(s, fs, M, N);
                                    m = melfb(20, N, fs);
                                    z = m * abs(frames(1:n2, :)).^2;
                                    tm = length(s) / fs;
                                    subplot(121)
                                    imagesc([0 tm], [0 fs/2], abs(frames(1:n2, :)).^2), axis xy;
                                    title('Power Spectrum unmodified');
                                    xlabel('Time [s]');
                                    ylabel('Frequency [Hz]');
                                    colorbar;
                                    subplot(122)
                                    imagesc([0 tm], [0 20], z), axis xy;
                                    title('Power Spectrum modified through Mel Cepstrum filter');
                                    xlabel('Time [s]');
                                    ylabel('Number of Filter in Filter Bank');
                                    colorbar;
                                else
                                    warndlg('Invalid Word ','Warning');
                                end
                            end
                        end
                    end
                    close all;

                else
                    warndlg('Database is empty.',' Warning ')
                end
            end

        end
    end
    %----------------------------------------------------------------------
\end{verbatim}


\subsection*{7: 2D plot of accustic vectors}

\begin{verbatim}
    % To inspect the acoustic space (MFCC vectors) we can pick any two
    % dimensions (say the 5th and the 6th) and plot the data points in a 2D
    % plane. Use acoustic vectors of two different speakers and plot data
    % points in two different colours. Do the data regions from the two
    % speakers overlap each other? Are they in clusters?
    %
    % Mostly the two areas overlap. But certain regions seem to be used
    % exclusively by one or the other speaker. This is what will allow us to
    % distinguish the different speakers.
    % The points don't form actual clusters, but there are areas where the
    % density of points is higher.
    % Note: This is only a two dimensional plot. The actual vector
    % contains 20 dimensions.

    if ch==7
        disp('>  7: 2D plot of accustic vectors');

        ch23=0;
        while ch23~=4
            ch23=menu('Select Folder','Train','Test','User','Exit');

            if ch23==1
                poss3=3;
                ch3=0;
                while ch3~=poss3
                    ch3=menu('2D plot of accustic vectors representation : ','1. One Signal',...
                        '2. Two Signal','3. Exit');

                    if ch3==1
                        ch31=0;
                        while ch31~=9
                            ch31=menu('2D plot of accustic vectors for :','Signal 1','Signal 2','Signal 3',...
                                'Signal 4','Signal 5','Signal 6','Signal 7','Signal 8','Exit');
                            if ch31~=9
                                plot(ctr{ch31}(5, :), ctr{ch31}(6, :), 'or');
                                xlabel('5th Dimension');
                                ylabel('6th Dimension');
                                st=sprintf('Signal %d ',ch31);
                                legend(st);
                                title('2D plot of accoustic vectors');
                            end
                        end
                        close all;
                    end

                    if ch3==2
                        ch32=0;
                        while ch32~=8
                            ch32=menu('2D plot of accustic vectors for :','Signal 1 & Signal 2',...
                                'Signal 2 & Signal 3','Signal 3 & Signal 4','Signal 4 & Signal 5',...
                                'Signal 5 & Signal 6','Signal 6 & Signal 7','Signal 7 & Signal 8','Exit');
                            if ch32~=8
                                plot(ctr{ch32}(5, :), ctr{ch32}(6, :), 'or');
                                hold on;
                                plot(ctr{ch32+1}(5, :), ctr{ch32+1}(6, :), 'xb');
                                xlabel('5th Dimension');
                                ylabel('6th Dimension');
                                st=sprintf('Signal %d,',ch32);
                                st1=sprintf('Signal %d', (ch32+1) );
                                legend(st,st1);
                                title('2D plot of accoustic vectors');
                                hold off
                            end
                        end
                    end
                    close all
                end
            end

            if ch23==2
                poss3=3;
                ch3=0;
                while ch3~=poss3
                    ch3=menu('2D plot of accustic vectors representation : ','1. One Signal',...
                        '2. Two Signal','3. Exit');

                    if ch3==1
                        ch31=0;
                        while ch31~=9
                            ch31=menu('2D plot of accustic vectors for :','Signal 1','Signal 2','Signal 3',...
                                'Signal 4','Signal 5','Signal 6','Signal 7','Signal 8','Exit');
                            if ch31~=9
                                plot(cte{ch31}(5, :), cte{ch31}(6, :), 'or');
                                xlabel('5th Dimension');
                                ylabel('6th Dimension');
                                st=sprintf('Signal %d ',ch31);
                                legend(st);
                                title('2D plot of accoustic vectors');
                            end
                        end
                        close all;
                    end

                    if ch3==2
                        ch32=0;
                        while ch32~=8
                            ch32=menu('2D plot of accustic vectors for :','Signal 1 & Signal 2',...
                                'Signal 2 & Signal 3','Signal 3 & Signal 4','Signal 4 & Signal 5',...
                                'Signal 5 & Signal 6','Signal 6 & Signal 7','Signal 7 & Signal 8','Exit');

                            if ch32~=8
                                plot(cte{ch32}(5, :), cte{ch32}(6, :), 'or');
                                hold on;
                                plot(cte{ch32+1}(5, :), cte{ch32+1}(6, :), 'xb');
                                xlabel('5th Dimension');
                                ylabel('6th Dimension');
                                st=sprintf('Signal %d,',ch32);
                                st1=sprintf('Signal %d', (ch32+1) );
                                legend(st,st1);
                                title('2D plot of accoustic vectors');
                                hold off
                            end
                        end
                    end
                    close all
                end
            end

            if ch23==3

                if (exist('sound_database.dat','file')==2)
                    load('sound_database.dat','-mat');

                    ch32=0;
                    while ch32 ~=2
                        ch32=menu('Database Information','Database','Exit');

                        if ch32==1
                            st=strcat('Sound Database has : #',num2str(sound_number),'words. Enter a database number : #');
                            prompt = {st};
                            dlg_title = 'Database Information';
                            num_lines = 1;
                            def = {'1'};
                            options.Resize='on';
                            options.WindowStyle='normal';
                            options.Interpreter='tex';
                            an = inputdlg(prompt,dlg_title,num_lines,def);
                            an=cell2mat(an);
                            a=str2double(an);

                            if (isempty(an))

                            else

                                if (a <= sound_number)
                                    st=strcat('u',num2str(an));
                                    [s fs]=wavread(st);
                                    c = mfcc(s, fs);
                                    plot(c(5, :), c(6, :), 'or');
                                    xlabel('5th Dimension');
                                    ylabel('6th Dimension');
                                    st1=sprintf('Signal %s.wav',st);
                                    legend(st1);
                                    title('2D plot of accoustic vectors');
                                else
                                    warndlg('Invalid Word ','Warning');
                                end
                            end
                        end
                    end
                    close all;

                else
                    warndlg('Database is empty.',' Warning ')
                end
            end

        end
    end
    %----------------------------------------------------------------------
\end{verbatim}


\subsection*{8: Plot of the 2D trained VQ codewords}

\begin{verbatim}
    % Plot the data points of the trained VQ codeword using the two dimensions
    % previously used for the plot of the last question.
    % The example in previous figure is an ideal case for teaching. In reality
    % however the clusters cannot be that easily distinguished. But again, this
    % is a two dimensional representation of a 20 dimensions space.

    if ch==8
        disp('>  8: Plot of the 2D trained VQ codewords');

        ch23=0;
        while ch23~=4
            ch23=menu('Select Folder','Train','Test','User','Exit');

            if ch23==1
                poss3=3;
                ch3=0;
                while ch3~=poss3
                    ch3=menu('2D plot of accustic vectors representation : ','1. One Signal',...
                        '2. Two Signal','3. Exit');

                    if ch3==1
                        ch31=0;
                        while ch31~=9
                            ch31=menu('2D plot of accustic vectors for :','Signal 1','Signal 2','Signal 3',...
                                'Signal 4','Signal 5','Signal 6','Signal 7','Signal 8','Exit');

                            if ch31~=9
                                plot(ctr{ch31}(5, :), ctr{ch31}(6, :), 'xr')
                                hold on
                                plot(dtr{ch31}(5, :), dtr{ch31}(6, :), 'vk')
                                xlabel('5th Dimension');
                                ylabel('6th Dimension');
                                st=sprintf('Speaker %d',ch31);
                                st1=sprintf('Codebook %d', (ch31) );
                                legend(st,st1);
                                title('2D plot of accoustic vectors');
                                hold off
                            end
                        end
                        close all
                    end

                    if ch3==2
                        ch32=0;
                        while ch32~=8
                            ch32=menu('2D plot of accustic vectors for :','Signal 1 & Signal 2',...
                                'Signal 2 & Signal 3','Signal 3 & Signal 4','Signal 4 & Signal 5',...
                                'Signal 5 & Signal 6','Signal 6 & Signal 7','Signal 7 & Signal 8','Exit');

                            if ch32~=8
                                plot(ctr{ch32}(5, :), ctr{ch32}(6, :), 'xr')
                                hold on
                                plot(dtr{ch32}(5, :), dtr{ch32}(6, :), 'vk')
                                plot(ctr{ch32+1}(5, :), ctr{ch32+1}(6, :), 'xb')
                                plot(dtr{ch32+1}(5, :), dtr{ch32+1}(6, :), '+k')
                                xlabel('5th Dimension');
                                ylabel('6th Dimension');
                                st=sprintf('Speaker %d',ch32);
                                st1=sprintf('Codebook %d',ch32 );
                                st2=sprintf('Speaker %d',(ch32+1) );
                                st3=sprintf('Codebook %d', (ch32+1) );
                                legend(st,st1,st2,st3);
                                title('2D plot of accoustic vectors');
                                hold off
                            end
                        end
                    end
                    close all
                end
            end

            if ch23==2
                poss3=3;
                ch3=0;
                while ch3~=poss3
                    ch3=menu('2D plot of accustic vectors representation : ','1. One Signal',...
                        '2. Two Signal','3. Exit');

                    if ch3==1
                        ch31=0;
                        while ch31~=9
                            ch31=menu('2D plot of accustic vectors for :','Signal 1','Signal 2','Signal 3',...
                                'Signal 4','Signal 5','Signal 6','Signal 7','Signal 8','Exit');

                            if ch31~=9
                                plot(cte{ch31}(5, :), cte{ch31}(6, :), 'xr')
                                hold on
                                plot(dte{ch31}(5, :), dte{ch31}(6, :), 'vk')
                                xlabel('5th Dimension');
                                ylabel('6th Dimension');
                                st=sprintf('Speaker %d',ch31);
                                st1=sprintf('Codebook %d', (ch31) );
                                legend(st,st1);
                                title('2D plot of accoustic vectors');
                                hold off
                            end
                        end
                        close all
                    end

                    if ch3==2
                        ch32=0;
                        while ch32~=8
                            ch32=menu('2D plot of accustic vectors for :','Signal 1 & Signal 2',...
                                'Signal 2 & Signal 3','Signal 3 & Signal 4','Signal 4 & Signal 5',...
                                'Signal 5 & Signal 6','Signal 6 & Signal 7','Signal 7 & Signal 8','Exit');

                            if ch32~=8
                                plot(cte{ch32}(5, :), cte{ch32}(6, :), 'xr')
                                hold on
                                plot(dte{ch32}(5, :), dte{ch32}(6, :), 'vk')
                                plot(cte{ch32+1}(5, :), cte{ch32+1}(6, :), 'xb')
                                plot(dte{ch32+1}(5, :), dte{ch32+1}(6, :), '+k')
                                xlabel('5th Dimension');
                                ylabel('6th Dimension');
                                st=sprintf('Speaker %d',ch32);
                                st1=sprintf('Codebook %d',ch32 );
                                st2=sprintf('Speaker %d', (ch32+1) );
                                st3=sprintf('Codebook %d', (ch32+1) );
                                legend(st,st1,st2,st3);
                                title('2D plot of accoustic vectors');
                                hold off
                            end
                        end
                    end
                    close all
                end
            end

            if ch23==3

                if (exist('sound_database.dat','file')==2)
                    load('sound_database.dat','-mat');

                    ch32=0;
                    while ch32 ~=2
                        ch32=menu('Database Information','Database','Exit');

                        if ch32==1
                            st=strcat('Sound Database has : #',num2str(sound_number),'words. Enter a database number : #');
                            prompt = {st};
                            dlg_title = 'Database Information';
                            num_lines = 1;
                            def = {'1'};
                            options.Resize='on';
                            options.WindowStyle='normal';
                            options.Interpreter='tex';
                            an = inputdlg(prompt,dlg_title,num_lines,def);
                            an=cell2mat(an);
                            a=str2double(an);

                            if (isempty(an))

                            else

                                if (a <= sound_number)
                                    st=strcat('u',num2str(an));
                                    [s fs]=wavread(st);
                                    c = mfcc(s, fs);
                                    d = vqlbg(c, 16);
                                    plot(c(5, :), c(6, :), 'xr');
                                    hold on
                                    plot(d(5, :), d(6, :), 'vk');
                                    xlabel('5th Dimension');
                                    ylabel('6th Dimension');
                                    st1=sprintf('Speaker %s',st);
                                    st2=sprintf('Codebook %s',st);
                                    legend(st1,st2);
                                    title('2D plot of accoustic vectors');
                                    hold off
                                else
                                    warndlg('Invalid Word ','Warning');
                                end
                            end
                        end
                    end
                    close all;

                else
                    warndlg('Database is empty.',' Warning ')
                end
            end

        end
    end
    %----------------------------------------------------------------------
\end{verbatim}


\subsection*{9: Recognition rate of the computer}

\begin{verbatim}
    % What is the recognition rate that our system can perform? Compare this
    % with the human's performances. Observe as well if the system makes errors
    % on the speech samples. If so, re -listen to the speech files and try to
    % come up with some explanations.
    % We used the following commands to test the system:
    %
    % The result is:
    % Our system was able to recognize 8 out of 8 speakers. This is an error
    % rate of 0.00%. The recognition rate of our system is much better than the
    % one of a human's recognition rate However you must be aware that this
    % test is not really representative of the computer's efficiency to
    % recognize voices because we only tested on 8 persons, with only one
    % training session and with only one word.
    %
    % Speaker 1 matches with speaker 1
    % Speaker 2 matches with speaker 2
    % Speaker 3 matches with speaker 3
    % Speaker 4 matches with speaker 4
    % Speaker 5 matches with speaker 5
    % Speaker 6 matches with speaker 6
    % Speaker 7 matches with speaker 7
    % Speaker 8 matches with speaker 8

    if ch==9
        disp('> 9: Recognition rate of the computer')
        %----------------------------------------------------------------------
\end{verbatim}


\subsection*{9.1 Loading from Test Folder for Comparison}

\begin{par}
All 8 samples data values are loaded in file sounddatabase1.dat.
\end{par} \vspace{1em}
\begin{verbatim}
        for sound_number = 1: 8
            if size(ste{sound_number},2)==2
                ste{sound_number}=ste{sound_number}(:,1);
            end
            ste{sound_number} = double( ste{sound_number} );
            data{sound_number,1} = ste{sound_number};
            data{sound_number,2} = sound_number;
            st = sprintf('s%d.wav',sound_number);
            data{sound_number,3} = st;
            data{sound_number,4} = 'Test';
            fs=fste{sound_number}; %#ok<NASGU>
            nb=nbte{sound_number}; %#ok<NASGU>
            if sound_number == 1;
                save('sound_database1.dat','data','sound_number','fs','nb');
            else
                save('sound_database1.dat','data','sound_number','fs','nb','-append');
            end
        end

        disp(' ');
        disp('Sounds From TEST added to database for comparison');
        disp(' ');
        %----------------------------------------------------------------------
\end{verbatim}


\subsection*{9.2 Comparing one by one data from TRAIN FOLDER}

\begin{par}
From Train folder samples are tested one by one to database stored in file sounddatabase1.dat And matching result is given on the basis of minimum distortion distance
\end{par} \vspace{1em}
\begin{verbatim}
        disp('Comparing one by one data from TRAIN FOLDER');
        disp(' ');
        load('sound_database1.dat','-mat');

        for classe = 1:8
            st = sprintf('Train\\S%d.wav to be compared',classe);
            disp(st);
            pause(0.5);
            if size(str{classe},2)==2
                str{classe}=str{classe}(:,1);
            end
            str{classe} = double(str{classe});
            %----- code for speaker recognition -------
            disp('MFCC cofficients computation and VQ codebook training in progress...');
            % Number of centroids required
            k =16;
            for ii=1:sound_number
                % Compute MFCC cofficients for each sound present in database
                v = mfcc(data{ii,1}, fstr{classe});
                % Train VQ codebook
                code{ii} = vqlbg(v, k);
            end
            disp('Completed.');
            disp(' ');
            % Compute MFCC coefficients for input sound
            v = mfcc(str{classe},fstr{classe});
            % Current distance and sound ID initialization
            distmin = Inf;
            k1 = 0;
            for ii=1:sound_number
                d = disteu(v, code{ii});
                dist = sum(min(d,[],2)) / size(d,1);
                if dist < distmin
                    distmin = dist;
                    k1 = ii;
                end
            end
            min_index = k1;
            speech_id = data{min_index,2};
            %-----------------------------------------
            disp('Matching sound:');
            disp(' ');
            message=strcat('File:',data{min_index,3});
            disp(message);
            message=strcat('Location:',data{min_index,4});
            disp(message);
            message = strcat('Recognized speaker ID: ',num2str(speech_id));
            disp(message);
            disp(' ');
        end

        disp('                                                             ');
        pause(0.5)
        msgbox('This prototype is 100% efficient in recognising these 8 different stored sounds in TEST and TRAIN folders. ','Success','help');
    end
    %----------------------------------------------------------------------
\end{verbatim}


\subsection*{9.3 Note: Test with other speech files}

\begin{verbatim}
    % You can also test the system with your own speech files. Use the Windows
    % program Sound Recorder to record your own voice and your friends' voices
    % Each new speaker needs to pro vide one speech file for training and one
    % for testing. Can the system recognize your voice? Enjoy.
    %----------------------------------------------------------------------
\end{verbatim}


\subsection*{10: Voice Recognition by letting user enter into database and then compare}

\begin{verbatim}
    %Voice recognition is the process of automatically recognizing who is
    %speaking on the basis of individual information included in speech waves.
    %
    %This technique makes it possible to use the speaker's voice to verify
    %their identity and control access to services such as voice dialing,
    %telephone shopping, database access services, information
    %services, voice mail, and remote access to computers.



    if ch==10
        disp('> 10: Test with other speech files')
        msgbox('P.S. This prototype is for secondary security usage.','NOTE','help');
        pause(2);
        msgbox('Kindly Note this works for the stored databases only. This means that you can add sounds to the database by users and Recognition will be done for the users entered. ','NOTE','help')
        pause(2);

        chos=0;
        possibility=5;
        while chos~=possibility,
            chos=menu('Speaker Recognition System','Add a new sound from microphone',...
                'Speaker recognition from microphone',...
                'Database Info','Delete database','Exit');

            %----------------------------------------------------------------------
\end{verbatim}


\subsection*{10.1 Add a new sound from microphone}

\begin{verbatim}
            if chos==1

                if (exist('sound_database.dat','file')==2)
                    load('sound_database.dat','-mat');
                    classe = input('Insert a class number (sound ID) that will be used for recognition:');
                    message=('The following parameters will be used during recording:');
                    disp(message);
                    message=strcat('Sampling frequency',num2str(samplingfrequency));
                    disp(message);
                    message=strcat('Bits per sample',num2str(samplingbits));
                    disp(message);
                    durata=input('Insert the duration of the recording (in seconds):');
                    micrecorder = audiorecorder(samplingfrequency,samplingbits,1);
                    disp('Now, speak into microphone...');
                    record(micrecorder,durata);

                    while (isrecording(micrecorder)==1)
                        disp('Recording...');
                        pause(0.5);
                    end
                    disp('Recording stopped.');
                    y1 = getaudiodata(micrecorder);
                    y = getaudiodata(micrecorder, 'uint8');

                    if size(y,2)==2
                        y=y(:,1);
                    end
                    y = double(y);
                    sound_number = sound_number+1;
                    data{sound_number,1} = y;
                    data{sound_number,2} = classe;
                    data{sound_number,3} = 'Microphone';
                    data{sound_number,4} = 'Microphone';
                    st=strcat('u',num2str(sound_number));
                    wavwrite(y1,samplingfrequency,samplingbits,st)
                    save('sound_database.dat','data','sound_number','-append');
                    msgbox('Sound added to database','Database result','help');
                    disp('Sound added to database');

                else
                    classe = input('Insert a class number (sound ID) that will be used for recognition:');
                    durata = input('Insert the duration of the recording (in seconds):');
                    samplingfrequency = input('Insert the sampling frequency (22050 recommended):');
                    samplingbits = input('Insert the number of bits per sample (8 recommended):');
                    micrecorder = audiorecorder(samplingfrequency,samplingbits,1);
                    disp('Now, speak into microphone...');
                    record(micrecorder,durata);

                    while (isrecording(micrecorder)==1)
                        disp('Recording...');
                        pause(0.5);
                    end
                    disp('Recording stopped.');
                    y1 = getaudiodata(micrecorder);
                    y = getaudiodata(micrecorder, 'uint8');

                    if size(y,2)==2
                        y=y(:,1);
                    end
                    y = double(y);
                    sound_number = 1;
                    data{sound_number,1} = y;
                    data{sound_number,2} = classe;
                    data{sound_number,3} = 'Microphone';
                    data{sound_number,4} = 'Microphone';
                    st=strcat('u',num2str(sound_number));
                    wavwrite(y1,samplingfrequency,samplingbits,st)
                    save('sound_database.dat','data','sound_number','samplingfrequency','samplingbits');
                    msgbox('Sound added to database','Database result','help');
                    disp('Sound added to database');
                end
            end

            %----------------------------------------------------------------------
\end{verbatim}


\subsection*{10.2 Voice Recognition from microphone}

\begin{verbatim}
            if chos==2

                if (exist('sound_database.dat','file')==2)
                    load('sound_database.dat','-mat');
                    Fs = samplingfrequency;
                    durata = input('Insert the duration of the recording (in seconds):');
                    micrecorder = audiorecorder(samplingfrequency,samplingbits,1);
                    disp('Now, speak into microphone...');
                    record(micrecorder,durata);

                    while (isrecording(micrecorder)==1)
                        disp('Recording...');
                        pause(0.5);
                    end
                    disp('Recording stopped.');
                    y = getaudiodata(micrecorder);
                    st='v';
                    wavwrite(y,samplingfrequency,samplingbits,st);
                    y = getaudiodata(micrecorder, 'uint8');
                    % if the input sound is not mono

                    if size(y,2)==2
                        y=y(:,1);
                    end
                    y = double(y);
                    %----- code for speaker recognition -------
                    disp('MFCC cofficients computation and VQ codebook training in progress...');
                    disp(' ');
                    % Number of centroids required
                    k =16;

                    for ii=1:sound_number
                        % Compute MFCC cofficients for each sound present in database
                        v = mfcc(data{ii,1}, Fs);
                        % Train VQ codebook
                        code{ii} = vqlbg(v, k);
                        disp('...');
                    end
                    disp('Completed.');
                    % Compute MFCC coefficients for input sound
                    v = mfcc(y,Fs);
                    % Current distance and sound ID initialization
                    distmin = Inf;
                    k1 = 0;

                    for ii=1:sound_number
                        d = disteu(v, code{ii});
                        dist = sum(min(d,[],2)) / size(d,1);
                        message=strcat('For User #',num2str(ii),' Dist : ',num2str(dist));
                        disp(message);

                        if dist < distmin
                            distmin = dist;
                            k1 = ii;
                            message=strcat(' and Distmin : ',num2str(distmin));
                            disp(message);
                        end
                    end

                    if distmin < ronaldo
                        min_index = k1;
                        speech_id = data{min_index,2};
                        %-----------------------------------------
                        disp('Matching sound:');
                        message=strcat('File:',data{min_index,3});
                        disp(message);
                        message=strcat('Location:',data{min_index,4});
                        disp(message);
                        message = strcat('Recognized speaker ID: ',num2str(speech_id));
                        disp(message);
                        msgbox(message,'Matching result','help');

                        ch3=0;
                        while ch3~=3
                            ch3=menu('Matched result verification:','Recognized Sound','Recorded sound','Exit');

                            if ch3==1
                                st=strcat('u',num2str(speech_id));
                                [s fs nb]=wavread(st);
                                p=audioplayer(s,fs,nb);
                                play(p);
                            end

                            if ch3==2
                                [s fs nb]=wavread('v');
                                p=audioplayer(s,fs,nb);
                                play(p);
                            end
                        end

                    else
                        warndlg('Wrong User . No matching Result.',' Warning ')
                    end
                else
                    warndlg('Database is empty. No matching is possible.',' Warning ')
                end
            end
            %----------------------------------------------------------------------
\end{verbatim}


\subsection*{10.3 Database Info}

\begin{verbatim}
            if chos==3

                if (exist('sound_database.dat','file')==2)
                    load('sound_database.dat','-mat');
                    message=strcat('Database has #',num2str(sound_number),'words:');
                    disp(message);
                    disp(' ');

                    for ii=1:sound_number
                        message=strcat('Location:',data{ii,3});
                        disp(message);
                        message=strcat('File:',data{ii,4});
                        disp(message);
                        message=strcat('Sound ID:',num2str(data{ii,2}));
                        disp(message);
                        disp('-');
                    end

                    ch32=0;
                    while ch32 ~=2
                        ch32=menu('Database Information','Database','Exit');

                        if ch32==1
                            st=strcat('Sound Database has : #',num2str(sound_number),'words. Enter a database number : #');
                            prompt = {st};
                            dlg_title = 'Database Information';
                            num_lines = 1;
                            def = {'1'};
                            options.Resize='on';
                            options.WindowStyle='normal';
                            options.Interpreter='tex';
                            an = inputdlg(prompt,dlg_title,num_lines,def);
                            an=cell2mat(an);
                            a=str2double(an);

                            if (isempty(an))

                            else

                                if (a <= sound_number)
                                    st=strcat('u',num2str(an));
                                    [s fs nb]=wavread(st);
                                    p=audioplayer(s,fs,nb);
                                    play(p);
                                else
                                    warndlg('Invalid Word ','Warning');
                                end
                            end
                        end
                    end

                else
                    warndlg('Database is empty.',' Warning ')
                end
            end
            %----------------------------------------------------------------------
\end{verbatim}


\subsection*{10.4 Delete database}

\begin{verbatim}
            if chos==4
                %clc;
                close all;

                if (exist('sound_database.dat','file')==2)
                    button = questdlg('Do you really want to remove the Database?');

                    if strcmp(button,'Yes')
                        load('sound_database.dat','-mat');

                        for ii=1:sound_number
                            st=strcat('u',num2str(ii),'.wav');
                            delete(st);
                        end

                        if (exist('v.wav','file')==2)
                            delete('v.wav');
                        end

                        delete('sound_database.dat');
                        msgbox('Database was succesfully removed from the current directory.','Database removed','help');
                    end

                else
                    warndlg('Database is empty.',' Warning ')
                end
            end
        end
    end

end
close all;
msgbox('Kindly motivate our efforts. Feel free to provide valuable feedback.','Thank You','help');
end
%--------------------------------------------------------------------------
\end{verbatim}


\subsection*{blockFrames Function}

\begin{par}
blockFrames: Puts the signal into frames
\end{par} \vspace{1em}
\begin{par}
Inputs: s contains the signal to analize fs is the sampling rate of the signal m is the distance between the beginnings of two frames n is the number of samples per frame
\end{par} \vspace{1em}
\begin{par}
Output: M3 is a matrix containing all the frames
\end{par} \vspace{1em}
\begin{verbatim}
function M3 = blockFrames(s, fs, m, n)
l = length(s);
nbFrame = floor((l - n) / m) + 1;
for i = 1:n
    for j = 1:nbFrame
        M(i, j) = s(((j - 1) * m) + i); %#ok<AGROW>
    end
end
h = hamming(n);
M2 = diag(h) * M;
for i = 1:nbFrame
    M3(:, i) = fft(M2(:, i)); %#ok<AGROW>
end
end
%--------------------------------------------------------------------------
\end{verbatim}


\subsection*{MFCC Function}

\begin{par}
MFCC
\end{par} \vspace{1em}
\begin{par}
Inputs: s contains the signal to analize fs is the sampling rate of the signal
\end{par} \vspace{1em}
\begin{par}
Output: r contains the transformed signal
\end{par} \vspace{1em}
\begin{verbatim}
function r = mfcc(s, fs)
m = 100;
n = 256;
frame=blockFrames(s, fs, m, n);
m = melfb(20, n, fs);
n2 = 1 + floor(n / 2);
z = m * abs(frame(1:n2, :)).^2;
r = dct(log(z));
end
%--------------------------------------------------------------------------
\end{verbatim}


\subsection*{VQLBG Vector quantization using the Linde-Buzo-Gray algorithm}

\begin{par}
VQLBG Vector quantization using the Linde-Buzo-Gray algorithm
\end{par} \vspace{1em}
\begin{par}
Inputs: d contains training data vectors (one per column) k is number of centroids required
\end{par} \vspace{1em}
\begin{par}
Output: r contains the result VQ codebook (k columns, one for each  centroids)
\end{par} \vspace{1em}
\begin{verbatim}
function r = vqlbg(d,k)
e = .01;
r = mean(d, 2);
dpr = 10000;
for i = 1:log2(k)
    r = [r*(1+e), r*(1-e)];
    while (1 == 1)
        z = disteu(d, r);
        [m,ind] = min(z, [], 2);
        t = 0;
        for j = 1:2^i
            r(:, j) = mean(d(:, find(ind == j)), 2); %#ok<FNDSB>
            x = disteu(d(:, find(ind == j)), r(:, j)); %#ok<FNDSB>
            for q = 1:length(x)
                t = t + x(q);
            end
        end
        if (((dpr - t)/t) < e)
            break;
        else
            dpr = t;
        end
    end
end
end
%--------------------------------------------------------------------------
\end{verbatim}



\end{document}
    
